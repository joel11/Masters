\documentclass[a4paper,latin]{paper} 
\usepackage[english]{babel}
\usepackage[textwidth=3cm,margin=1.2cm,columnsep=1cm,bottom=2cm, top=1.8cm]{geometry}
\usepackage{graphicx}
\usepackage{xcolor}
\usepackage{booktabs}
\usepackage{tcolorbox}
%\usepackage{tabular}

%\setlength\textwidth{\dimexpr (3in -1in/16)*2 + 3in/8\relax}
%\setlength\columnsep{\dimexpr 3in/8\relax}

\usepackage{color}
\usepackage[linktocpage,colorlinks=true,linkcolor= {red!50!black}, urlcolor=black, citecolor=blue!90, pdfborder={2 1 0}]{hyperref}
\usepackage{hyperref}
\usepackage{amsmath}
\usepackage{float}
\usepackage{fancyhdr}
\usepackage{refcount}
\usepackage{longtable}
%\usepackage{fancyvrb}
\usepackage{array}
\usepackage{tabularx}
 \usepackage{colortbl} 
\usepackage{subcaption}
\usepackage{caption}
\usepackage[numbib,nottoc]{tocbibind}

\sectionfont{\large\sf\bfseries\color{black!70!blue}} 
%\renewcommand\keywordname{Clavem verborum}


\title{Online non-linear prediction of financial time-series patterns}
%\hfill\includegraphics[height=2cm]{/home/fran/logo}}
\author{Mr Joel da Costa \\ Supervisor: Prof. Tim Gebbie} 
\institution{University of Cape Town \\ MSc. Advanced Analytics (STA5004W) Research Proposal}

\begin{document} 
\twocolumn[\maketitle 
\hrule 
\vspace{-2mm}
\begin{abstract} 
We consider pattern prediction of financial time-series data. The algorithm framework and workflow is developed
and proved on daily sampled OHLCV (open-high-low-close-volume) time-series data for JSE equity markets. The
input patterns are based on pre-processing using FFT data interpolation for missing and outlier data points. The
input data vectors are equal size data windows pre-processed into a sequence of daily, weekly and monthly
sampled feature measurement changes (here log feature fluctuations). The data processing is split into at offline
batch processed step where data is compressed using an autoencoder via unsupervised learning, and then batch
supervised learning is carried out using the data-compression algorithm with the output being a pattern sequence
of measured time-series feature fluctuations (log differenced data) in the future (ex-post) from the training and
validation data. The historical simulation is then run using an online neutral network initialised with the weights from
the offline training and validation step. The historical simulation is then considered in terms of test for statistical
arbitrage and a simple correction for transaction costs is considered.\end{abstract}



\begin{keywords}
​online learning, neural network, pattern prediction, JSE, non-linear, financial 
​time series
\end{keywords}
\hrule\bigskip
]


\section{Hypotheses} 
\begin{itemize}
\item Primary Hypothesis: A online non-linear model can be trained to predict pattern 
matches in OHLCV securities data
\item Secondary Hypothesis: Preprocessing of data through an autoencoder can be 
shown to produce statistically significant improvements
\end{itemize}

\section{Literature Review} 
Technical analysis is a financial analysis practice that makes use of past price data in order to identify market structures, as well as forecast future price movements. The techniques are typically objective methodologies which rely solely on past market data (price and volume). They stand in contrast to fundamental analysis, where experts will consider a companies operations, management and future prospects in order to arrive at an evaluation. The basis of much technical analysis, originally developed through Dow Theory, is the belief that stock market prices will move directionally (upwards, downwards or sideways), and that past movements can be used to determine these trends \cite {murphy}.
\hfill \break \break
One of the primary methods in technical analysis is the use of charts in order to identify price patterns. These charts will be produced using the available market data and a known design, such as the popular candle-bar plot, which can then be compared to historical data to match it to a particular pattern. These patterns are thus indicative that the stock is likely to take on a particular price trend, or is in a particular state \cite {murphy}.  There is a certain amount of controversy around technical analysis, where many argue that it is contradictory to the random walk and weak form efficient market hypotheses, and as such is not valuable or useful \cite {griffioen}. The argument against this, is that technical analysis does not rely on past action to predict the the future, but is rather a measure of current trading, and how the market has reacted after similar patterns have occurred in the past \cite {kahn}. Further, even if the analysis is unable to effectively forecast future price trends, it can still be useful to exploit trading opportunities in the market \cite{schwager}. 
\hfill \break \break
With the advent of processing power becoming cheaply available, there has been an increase in research to adapt computing techniques to technical analysis. The breadth and superhuman speed in which systems are able to perform technical analysis far outstrips what was possible before, and as such they have become the focus of competitive performance for many market participants \cite {johnson}. To this end, there has been much research to apply machine learning algorithms to perform pattern recognition on stock price movements.
\hfill \break \break
Financial markets have been shown to be complex and adaptive systems, where the effects of interaction between participants can be highly non-linear \cite {arthur}. Complex and dynamic systems such as these may often exist at the 'order-disorder border' - they will generate certain non-random patterned and internal organisation, which can be assessed and identified, however they will also exhibit a certain amount of randomness in their behaviours, or 'chaos' \cite {crutchfield}. As a result, trying to identify these patterns and structures is a simultaneously reasonable and notoriously difficult goal. While it is often clear in hindsight that the patterns exist, the amount of noise and nonlinearity in the system can make prediction challenging.
\hfill \break \break
Fittingly then, neural networks have become a popular choice for modelling within the financial markets. Due to their structure, they are able to learn non-linear interactions between their inputs and outputs, with even early research showing their ability to achieve statistically significant results \cite {skabar}. They have been shown to be universal function approximators: given the correct data and internal structure, they are able to model any input and output relationship \cite {hornik}. Neural networks have been shown to be capable of generating trading signals which outperform traditional buy and hold strategies - simultaneously refuting the efficient market hypothesis, as well as demonstrating the networks capacity to capture the non-linear relationships of the market \cite {skabar}. 
\hfill \break \break
More recent work has lead to the development of convolutional neural networks, and more generally, deep belief networks. These take the traditional feed forward networks with single hidden layers, and extend both the layered architecture, as well as the processing between layers. Research such as ImageNet has shown this structure to be incredibly adept at pattern recognition \cite{krizhevsky}\cite{oord}. This structure has been further adapted to show effectiveness at predicting time series data which outperforms base neural models \cite {borovykh}. 
\hfill \break \break
Pang et al. have shown a novel approach in deep belief networks layering auto encoders prior to the neural networks (long short-term memory in this case) which can be effective, though the results leave room for improvements \cite {pang}. Bao et al. also showed that the use of stacked auto encoders in a deep belief network (LTSM again) lead to better performance than other neural network model without the auto encoders \cite {bao}. On a similar note, Hatami et al. were able to show that converting one dimensional time series data to a 2 dimensional recurrence plot was an effective way in improving prediction results for convolutional neural networks \cite {hatami}. Evidently, there is room and reason for investigation in how preprocessing of time series data is able to increase performance in neural networks.
\hfill \break \break
There is a certain disconnect in some of the approaches mentioned here and the practicality of implementing them in a financial context. Many machine learning and predictive models require extensive batch training, with the assumption that there is an substantial or full range of input covered. In reality, particularly in the realm of financial prediction, input is only partially available and further input will continue to arrive in the future. Models that are able to operate by updating themselves once more data becomes available are known as 'online' models \cite {albers}. These are naturally applicable in the realm of financial time series data. Some of the more extensive deep belief models, such as LTSM, can have time consuming training times \cite {bao}, which may negate their use for an online network.
\hfill \break \break
Suggested financial strategies are typically justified through backtests - a simulation of the investment strategy on historical data. In these simulations, 
the strategies performance over a set of data is recorded to determine the profits and losses it would have generated, as well as the more popular financial 
indicators such as the Sharpe ratio (used to quantify the return on risk). 
There is an important distinction here between the In-Sample (IS) and Out of Sample (OOS) datasets. 
Ideally, the algorithm is developed on the training, or IS data, and tested on the unseen validation data, or OOS data. 
Bailey et al. \cite {bailey} have discussed and shown that both over fitting and data snooping are rife in the literature on trading strategies. 
These issues within the backtesting simulations largely invalidate the results for practical purposes, as there’s no 
clear evidence of how the strategy would operate in the market. There is a clear parallel in the structuring of data 
for backtests and for training machine learning models. However, as Bailey et al. have pointed out, the methodology is not always applicable. 
There are difference in outputs (e.g. point forecasts versus trading signals), potential problems with overfitting other aspects of a trading strategy (e.g. entry or stop-loss thresholds), 
and a lack of accounting for the number of trials attempted in regression fitting. 
Bailey et al. later developed an effective and novel method ( combinatorially
symmetric cross-validation) of running backtest simulations 
in such a way that overfitting is unlikely to occur, as well as offering a 
probabilistic measure of whether it has occured or not \cite {bailey2}. 

\hfill \break \break
With consideration to the aspects raised above, this project will aim to 
implement an online, feed-forward neural network to recognise patterns in 
financial time series data, and so predict feature fluctuations. Feedforward 
networks will be used so as to make the online updating viable in terms of 
speed. Consideration will be given to the preprocessing of data and using auto-encoders in order to improve performance.
The model will then be trained and tested, using appropriate backtesting 
methodolgies (as per \cite {bailey2}), on OHLCV data from the JSE, with expansion to intraday data 
afterwards. 

\section{Aims and Objectives}  
\begin{itemize}
\item[1] Create a feature set artificially made up of time-series data slices with a bespoke ‘convolution’ setup 
\begin{itemize}
\item This will require data pre-processing that is fast and which can be both online and offline
\item FFT Interpolation will be used for missing data
\end{itemize}
\item[2] Compare daily sampled data from various difference equity classes
\item[3] Use dimensional reduction through an auto encoder to show performance 
improvements
\item[4] Create a library which is able to generate the full online non-linear prediction model through the use of auto-encode and an online neural network. This needs to be able to achieve n-order prediction both online and offline
\begin{itemize}
\item The typical basis of statistical learning will need to be considered: training, validation and testing
\end{itemize}
\item[5] Create backtesting module for the online model
\begin{itemize}
\item This will be novel if it is correctly implemented online step t to t+1 without any batch update and correctly accounts for cost
\end{itemize}
\item[6] Discuss the generalisation error, as well as in-sample vs out-sample errors and performance statistics of the backtesting results
\item[7] Consider the use of the online model for a profitable trading strategy
\begin{itemize}
\item Trading and transaction costs will need to be taken into account
\item Carry out a standard hypotheses test for a statistical arbitrage (test for positive probability of excess returns, vanishing variance and zero probability of gamblers ruin.)
\end{itemize}
\end{itemize}
\section{Data Requirements Specification}
\begin{itemize}
\item A surrogate dataset of one or two stocks will be used to start 
development
\item Thereafter a fuller set of data will need to be collected, from Bloomberg 
and Thomson-Reuters
\item Bloomgberg OHLCV daily data for 20 years will be considered in the stock 
combinations as detailed in Table 1
\item Thomson Reuters Tick History intraday data consisting of top-of-book and transaction updates for 
the same stocks as listed in Table 1
\begin{itemize}
\item Data will be processed to create 5 minute and 10 minute bars from the intraday data as well as volume time bars to be used as an input to the online learning algorithm
\end{itemize}
\item Synthetic data cases (Eg. Monte Carlo simulations) will also be considered in order to discuss issues encountered with in-sample versus out-of-sample backtesting
\begin{itemize}
\item Examples of such data cases would be where stocks are all 
increasing/decreasing over time, or both for a combinations of stocks.
\end{itemize}
\item As detailed in table 1, there are 5 primary stocks, each of which will be considered in various 
Stock, Equity Index and Bond Index pairs, as well as by themselves. E.g. For AGL, the following will 
be considered: AGL, AGL and BHP, AGL and ALSI40 and AGL and ALBI
\item The Daily TRI (Total Return Index) OHLCV for 20 years will be considered 
for all pairs, and the Intraday data will be considered for the Single and Stock 
pairs.)
\end{itemize}

\begin{table}[H]
\begin{tcolorbox}[tabularx*={\arrayrulewidth0.6mm}{X||X|X|X},fonttitle=\bfseries\large,fontupper=\normalsize\sffamily,
colback=white!10!white,colframe=black!40!,
coltitle=black,center title,toprule=1mm]
\centerline{\textbf{Stock}} &\centerline{\textbf{Stock Pair}} &  \centerline{\textbf{Equity Index}} &\centerline{\textbf{Bond Index}} \\\hline\hline

\vspace{0.1mm}  \centerline {AGL} &\vspace{0.1mm}  \centerline {BHP}  & \vspace{0.1mm} \centerline {ALSI40} &  \vspace{0.1mm}\centerline {ALBI}  \\\hline
\vspace{0.1mm}  \centerline {SBK} &\vspace{0.1mm}  \centerline {SNL}  & \vspace{0.1mm} \centerline {ALSI40} &  \vspace{0.1mm}\centerline {ALBI}  \\\hline
\vspace{0.1mm}  \centerline {SHF} &\vspace{0.1mm}  \centerline {RCH} & \vspace{0.1mm} \centerline {ALSI40} &  \vspace{0.1mm}\centerline {ALBI}  \\\hline
\vspace{0.1mm}  \centerline {WHL} &\vspace{0.1mm}  \centerline {SHP}  & \vspace{0.1mm} \centerline {ALSI40} &  \vspace{0.1mm}\centerline {ALBI} \\\hline
\vspace{0.1mm}  \centerline {MTN} &\vspace{0.1mm}  \centerline {VOD}  & \vspace{0.1mm} \centerline {ALSI40} &  \vspace{0.1mm}\centerline {ALBI}  \\\hline

\end{tcolorbox}
\caption{Stock Combinations}
\end{table}

\section{Data Science Workflow}
The project will be split up into several stages,  as per the processes that the 
data will go through. These are detailed below, and expanded on in \cite{technicaldoc}

\begin{itemize}
\item[1] FFT Interpolation:
This will process the raw OHLCV data, and inerpolate any missing points using FFT
\item[2]Data Processing:
Using the now complete dataset, this will layer multiple time slices of the data's feature fluctuations  
\item[3] Dimension Reduction:
This layer will perform dimension reduction on the time sliced data using an 
Autoencoder
\item[4] Offline Neural Network:
A standard ANN using data processed through the Autoencoder as input to predict 
n-step fluctuations
\item[5] Online Neural Network \& Backtesting module:
An online version of the offline neural network, validated using testing 
techniques similar to those detailed by \cite{bailey} \cite{bailey2}
\end{itemize}


\section{System Requirements Specification}

\begin{itemize}
\item Hardware: A macbook pro will be use for the crux of the development, with the following specs:  2,8 GHz Intel Core 
i7, 16 GB 2133 MHz LPDDR3, SSD
\item Software: Julia will be the primary programming language used to develop the project, though Python and 
R will also be used interchangeably for Exploratory Data Analysis (EDA), and as needed.  
There may be various public libraries used within all of these languages that 
are considered at the time
\item Each of the project stage deliverables will be done by following the 
process detailed in the steps below. These are detailed further in \cite{technicaldoc}
\begin{itemize}
\item[1]  Problem Definition
\item[2] Data Processing
\item[3] Data Exploration
\item[4] Baseline Modeling
\item[5] Secondary Modeling
\item[6] Testing
\item[7] Process Preparation

\end{itemize}

\end{itemize}

\section{Project Milestone Deliverables}
\begin{table}[H]
\begin{tcolorbox}[tabularx*={\arrayrulewidth0.6mm}{X||X|X},fonttitle=\bfseries\large,fontupper=\normalsize\sffamily,
colback=white!10!white,colframe=black!40!,
coltitle=black,center title,toprule=1mm]
\centerline{\textbf{Date}} & \centerline{\textbf{Description}} & \centerline{\textbf{Deliverable}} \\\hline\hline
%\vspace{0.1mm}  \centerline{10 Jul 17} &  dGFKD  & Basic Materials \\
 %                     &   $\boldsymbol{\cdot}$ Introduce parfors    & 
%\\\hline
\vspace{0.1mm}  \centerline{31/03} & Literature Review \& Learn Julia Basics  &  \\\hline
\vspace{0.1mm}  \centerline{21/04} & FFT Data Interpolation & FFT Library  \\\hline
\vspace{0.1mm}  \centerline{15/05} & Synthetic Data Generation & Synthetic Data Collection \\\hline
\vspace{0.1mm}  \centerline{31/05} & Data Preprocessing & Processing Library \\\hline
\vspace{0.1mm}  \centerline{21/06} & Dimensional Reduction & Auto Encoder class \\\hline
\vspace{0.1mm}  \centerline{10/07} & Finalise Data Collection & Final Dataset \\\hline
\vspace{0.1mm}  \centerline{05/08} & Offline Neural Network & Training Library, Model, Validation Test Outputs \\\hline
\vspace{0.1mm}  \centerline{31/08} & Online Neural Network & Process Library and Model \\\hline
\vspace{0.1mm}  \centerline{30/09} & Backtesting Module & Process Library and Statistical Test Results \\\hline
\vspace{0.1mm}  \centerline{31/10} & Testing on Exteded Datasets & Models and Statistical Test Results \\\hline
\vspace{0.1mm}  \centerline{30/11} & Dissertation Write Up and Revisions & Final Hand-in\\\hline

\end{tcolorbox}
\caption{Key dates and deliverables for research project}
\end{table}
\begin{thebibliography}{9}
\bibitem{murphy} 
John J. Murphy, Technical Analysis of the Financial Markets (New York Institute of Finance, 1999), pages 1-5,24-31.

\bibitem{griffioen} 
Griffioen, Gerwin A. W., Technical Analysis in Financial Markets (March 3, 2003).

\bibitem{kahn}
Kahn, Michael N. . Technical Analysis Plain and Simple: Charting the Markets in Your Language, Financial Times Press, Upper Saddle River, New Jersey, p. 9. ISBN 0-13-134597-4.(2006)

\bibitem {schwager}
Jack D. Schwager. Getting Started in Technical Analysis, Page 2 (1999)

\bibitem {johnson}
N. Johnson, G. Zhao, E. Hunsader, H. Qi, N. Johnson, J. Meng & Brian Tivnan (2013). Abrupt rise of new machine ecology beyond human response time. Sceintific Reports 3(2627). DOI: 10.1038/srep02627

\bibitem{arthur}
B. Arthur (1995) Complexity in Economics and Financial Markets, Complexity 1 (1), 20–25.

\bibitem {crutchfield}
J. P. Crutchfield, (2011), Between order and chaos, Nature Physics, Vol. 8, January 2012, 17-23

\bibitem {skabar}
 Skabar, Cloete, Networks, Financial Trading and the Efficient Markets Hypothesis (http://crpit.com/confpapers/CRPITV4Skabar.pdf)

\bibitem {hornik}
K. Hornik, Multilayer feed-forward networks are universal approximators, Neural Networks, vol 2, 1989

\bibitem {krizhevsky}
A. Krizhevsky, I. Sutskever, and G. E. Hinton, ImageNet Classification with Deep Convolutional Neural Networks, Advances in Neural Information Processing Systems 25, (2012), pp. 1097–1105.

\bibitem {oord}
Aaron van den Oord, WaveNet: A Generative Model For Raw Audio (19 September 2016) 
\url{https://arxiv.org/pdf/1609.03499.pdf}

\bibitem {borovykh}
A. Borovykh, Conditional Time Series Forecasting with Convolutional Neural 
Networks (Oct 18th 2017) \url {https://arxiv.org/pdf/1703.04691.pdf}

\bibitem {pang}
X. Pang, An innovative neural network approach for stock market prediction (12th Jan 2018) 
\url {https://link.springer.com/article/10.1007/s11227-017-2228-y}

\bibitem {bao}
W. Bao, A deep learning framework for financial time series using stacked autoencoders and long-short term memory 
(14 July 2017) \url {http://journals.plos.org/plosone/article?id=10.1371/journal.pone.0180944}

\bibitem {hatami}
Nima Hatami, Classification of Time-Series Images Using Deep Convolutional Neural Networks 
(7th Oct 2017) \url {https://arxiv.org/pdf/1710.00886.pdf}

\bibitem {albers}
S. Albers, Ser. B (2003). Online algorithms: a survey. Mathematical Programming 97(1). doi:10.1007/s10107-003-0436-0

\bibitem {bailey}
D. H. Bailey, J. M. Borwein, M. M. L´opez de Prado, & Q. J. Zhu (2014). Pseudo-Mathematics and Financial Charlatanism: The Effects of Backtest Overfitting on Out-of-Sample Performance. Notices of the AMS, 61(5), 458-471

\bibitem{bailey2}
D. H. Bailey, J. M. Borwein, M. M. L´opez de Prado, & Q. J. Zhu (2015). The Probability of Backtest Overfitting

\bibitem{technicaldoc}
J da Costa, 'Online non-linear prediction of financial time-series patterns, 
Technical document accompaniment (2018).

\end{thebibliography}
\end {document}


