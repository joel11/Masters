\documentclass[a4paper,latin]{paper} 
\usepackage[greek,english]{babel}
\usepackage{textgreek}
\usepackage[textwidth=3cm,margin=1.2cm,columnsep=1cm,bottom=2cm, top=1.8cm]{geometry}
\usepackage{graphicx}
\usepackage{xcolor}
\usepackage{booktabs}
\usepackage{tcolorbox}
%\usepackage{tabular}

%\setlength\textwidth{\dimexpr (3in -1in/16)*2 + 3in/8\relax}
%\setlength\columnsep{\dimexpr 3in/8\relax}

\usepackage{color}
\usepackage[linktocpage,colorlinks=true,linkcolor= {red!50!black}, urlcolor=black, citecolor=blue!90, pdfborder={2 1 0}]{hyperref}
\usepackage{hyperref}
\usepackage{amsmath}
\usepackage{float}
\usepackage{fancyhdr}
\usepackage{refcount}
\usepackage{longtable}
%\usepackage{fancyvrb}
\usepackage{array}
\usepackage{tabularx}
 \usepackage{colortbl} 
\usepackage{subcaption}
\usepackage{caption}
\usepackage[numbib,nottoc]{tocbibind}

\newcommand{\myol}[2][3]{{}\mkern#1mu\overline{\mkern-#1mu#2}}


\sectionfont{\large\sf\bfseries\color{black!70!blue}} 
%\renewcommand\keywordname{Clavem verborum}


\title{Literature Review}
%\hfill\includegraphics[height=2cm]{/home/fran/logo}}

\begin{document} 
\maketitle  
\hrule

\section{Technical Analysis}\label{TechnicalAnalysis}

Brief introduction to Technical Analysis and pattern matching.

\section{Neural Networks}

\subsection{Neural Network Algorithms and Structures}

An overview of the relevant algorithms, and early network structures.

\subsection{Deep Learning}

A brief overview of the developments in deep learning as applied to neural 
networks

\subsection{Neural Network Variations}

A brief overview of the variations of neural networks - e.g. perceptrons, RBM, 
feedforward, LTSM etc.

\subsection{Applicatons of Neural Networks in Finance}

A coverage of the relevant applications of neural networks in predicting prices.

\section{Stacked Autoencoders}

\subsection{High Dimensional Data Reduction}\label{HDDR}

As noted, machine learning techniques have been shown to be extremely effective at modelling non-linear inputs 
to outputs - neural networks have even been shown to be universal function approximators in this regard \cite{Hornik}. 
More traditional statistical models will typically process the available feature data to select the most significant 
features to be used in the model once it’s defined - evident in a processes such as subset selection \cite{Schaefer}. 
Machine learning techniques are no different in this regard, and feature data will typically be transformed to smaller 
observations of more significance prior to being used as input to a model, such as the neural networks described above.
\hfill \break 

Financial data, in line with the complex and dynamic system that it represents, is often of a very high dimensional 
nature, which offers opportunities through more sophisticated analysis, but also introduces the curses of
 dimensionality \cite{Donoho}. The increased dimensionality can result in higher processing complexities when needing to 
 do basic tasks such as estimating a covariance matrix (a commonplace necessity in finance), as well as increase 
 the risk of incorrect assumptions based on spurious variable collinearity \cite{Fan1}. Noise accumulation in high 
 dimensional data can create further problems, resulting in problems performing variable selection and ultimately 
 having a large impact on classification and regression models \cite{Fan2}.
\hfill \break 

Time series data can introduce its own set of challenges - there is often not enough data available to understand 
and predict the process \cite{Fama}, the time variable dependence creates complexity in how much past 
data to consider at any point, and the data is typically non-stationary \cite{Langkvist}. Thus, high dimensional time 
series data (which many financial problems focus on), require careful consideration on how to handle their inputs 
and analysis.
\hfill \break 

Deep learning techniques are a natural choice in this context, and much research has been done to show their 
(varying) efficacy on time series data. The most successful of these models have been ones which modify deep 
learning techniques to incorporate the temporal aspect of the data (e.g. Conditional Restricted Boltzmann 
Machines or Recurrent Neural Networks), rather than static, and those which have performed feature selection
 processes rather than operating on the raw data (e.g. Auto-encoders)  \cite{Langkvist}. 
 \hfill \break 
 
 
Two of the seminal pieces of research that have lead to the resurgence in machine learning and deep learning 
were the algorithms for training deep belief networks \cite{Hinton1}, as well as the usage of stacked auto-encoders
\cite{Ranzato1, Bengio1}. 

 
 \subsection{Deep Belief Networks}\label{DBN}
 
 Autoencoders were suggested by Hinton et. al as a method of transforming high dimensional 
 data to lower dimensional input vectors, in order to alleviate some of the problems detailed above, and increase 
 performance of deep belief networks \cite{Hinton2}.
\hfill \break 

One of the more prominent classical techniques for dimension reduction is principal components analysis (PCA), 
which uses linear algebra to find the directions of greatest variance, and represent the observation samples 
features along each of these directions, thus maximising the variational representation. Hinton et al. show that 
autoencoders are a nonlinear generalization of PCA. The structure and training algorithms of the autoencoder 
show it to be a specialised neural network - there is a multilayer encoder network which is able to transform to a 
lower dimension, and a symmetrical decoder network to recover the data from the code as represented in Figure \ref{figure-DBN-RBM}. As with 
neural networks, the gradient weights can be trained through the feedforward and backpropagation algorithms.  
\hfill \break 

The primary challenge presented here is the initial weighting of the networks - with large initial weights the 
autoencoder will often find a poor local minima, and with small initial weights the gradients are too small to 
effectively train deep layered networks. The critical suggestion by Hinton et al. was to used layered Restricted 
Boltzmann Machines (RBM) in order to initialise the weights. For each layer of the desired autoencoder, a RBM is 
formed and trained with the previous layer (or RBM) \cite{Hinton3}. Once all the layers have been 
trained in this way, they are mirrored to form the decoder network. This then forms the initial weights to be fine 
tuned further, as per the Fine-tuning step in \ref{figure-DBN-RBM}. They showed the deep autoencoder networks were significantly more 
effective than PCA or shallow autoencoders on multiple dataset types. 
 \hfill \break 
 
\begin{figure}
\centering \includegraphics[scale=0.5]{images/DBN-RBM-process.png}
 \caption{The Autoencoder training steps \cite{Hinton2}}
 \label{figure-DBN-RBM}
 \end{figure}
 
 \subsection {Stacked Denoising Autoencoders}\label{SDAE}

 The second important piece of work was the development of a denoising autoencoder (DAE), by Vincent  et al. \cite{Vincent}. 
 One of the problems identified in the DBN model (and those similar), is that if the encoder dimensions were too high, 
 it is likely that the encoder would learn a trivial encoding - essentially creating a “copy the input” model. The one way
  of tackling this issue is to constrain the representation with bottlenecks and sparse autoencoder layers, which can be seen in figure  \ref{figure-DBN-RBM}.
  \hfill \break 

Vincent et al. explore a very different approach to the problem, which was to develop an implementation of 
autoencoder which focused on partially corrupting the input, and so force the network to ‘denoise’ it. The theory 
here is based on two ideas - the first, is that a higher dimensional representation should be robust to partial 
corruption of the input data; and the second is that the denoising process will force model focus to shift to 
extracting useful features from the input.
 \hfill \break 

The algorithms and structures are largely the same as described for DBNs above, with the key difference 
being that the model is trained to reconstruct the original input, but only using a corrupted version of the input 
(where noise has been added to it), and so is forced to learn smarter feature mappings and extractions. 
The DAE suggested then is a stochastic variant of the autoencoder, which has the benefit of being able to 
implement higher dimensional representations without risking training of a trivial identity mapping. Notably, 
in the Stacked Denoising Autoencoder (SDAE) formation, only the initial input is corrupted (as opposed to the 
input from layer to layer). It was shown that the SDAE model outperformed previous AE and DBN networks on 
numerous benchmark datasets \cite{Vincent} . 

\subsection {Pre-training}

The methods described above follow a similar approach: greedy layer-wise unsupervised pre-training in order to 
determine initial weights, followed by supervised fine tuning to arrive at the final model. It is shown numerous times, 
that the pretraining process results in significant performance gains \cite{Vincent}. However it is not immediately apparent, 
given the nature of backpropagation algorithms and the like, why this is the case. Erhan et al. performed 
extensive empirical simulations in order to suggest an explanation to the mechanism of pretraining \cite{Erhan}.
 \hfill \break 

While their results were not entirely conclusive, they did lend themselves to a reasonable hypothesis: 
the unsupervised pre-training results in a form of regularization on the model - variance is minimized, and the 
bias introduced directs the model configuration towards a sample space that is effective for the unsupervised 
learning generalization optimisations.

\subsection {Time Series Applications}

The autoencoder papers reviewed so far in this section derive their results primarily from classification problems, 
and so do not necessarily account for the problems involved with time series as described in \ref{HDDR}. Due to 
the inherent difficulties with predictions in the financial system, it can sometimes be unclear if the shortcoming in 
results is due to this system complexity or if the methodologies used are unsuited for the purpose. In light of this 
it is worth pointing out that Stacked Autoencoder (SAE) implementations have been shown to be effective in many
 time series systems.  
 \hfill \break 

Lv et al.  implemented a deep learning SAE model using the methods described in \ref{SDAE} in order to 
predict traffic flow at various time intervals (15, 30, 45 and 60 minutes) - a problem not so structurally dissimilar 
from what will be presented in this paper \cite{Lv}. They were able to show that the deep SAE was able to offer prediction 
results which were both objectively good and also persistently outperformed the comparison models used 
(backpropagation neural network, random walk forecast, support vector machine and a radial basis function 
neural network).
\hfill \break 

In a review of unsupervised feature learning and deep learning methods on time series, Langkvist et al. noted that 
the use of autoencoders, either as a technique in themselves, or as an auxiliary technique to models 
such as convolutional neural networks, were able to offer performance increases in areas such as video analysis, 
motion capture data and bacteria identification \cite{Langkvist}.
 
\subsection{Financial Applications}

There have of course also been successful applications of stacked autoencoders and deep learning models in 
finance as well. Takeuchi et al. performed some earlier work showing the use of autoencoders when applied to a 
momentum trading strategy. They implemented an RBM pre-trained DBN as per \ref{DBN}, and assessed the 
networks classification performance for ordinary shares on NYSE, AMEX and Nasdaq. This showed that using a 
DBN network resulted in significant performance increases compared to the standard momentum strategy \cite{Takeuchi}.
\hfill \break 

Zhao et al. used SDAEs and combined them with the bootstrap aggregation ensemble method (‘bagging’) in a 
study of predicting the crude oil price. They compared the proposed model to a variety of benchmarks, including 
standard SAE, bagged and standard feedforward networks and SVRs. The results indicated that the SAE models 
were more accurate, with the bagged SAE model performing the best, though at a significant increase in 
computational costs in comparison to standard SAE \cite{Zhao}.
\hfill \break 

While much of the financial literature has focused on the use of RBM based models, Autoencoders and SAEs have 
recently been gaining popularity in performing feature reduction. Troiano et al. specifically investigate the use of 
different feature reduction models for trend prediction in finance \cite{Troiano}. In line with being primarily 
interested in the effect of feature reduction techniques, rather than the classification performance itself, only an 
SVM model was used to test results. Using various periods from historical S&P 500 data, they were able to show 
that AE outperformed the RBM model significantly in numerous accuracy measures, and was able to do so at a 
fraction of the training time.
\hfill \break 

Bao et al.  note that the research has been lacking with regards to whether SAEs should be used for 
financial prediction models or not \cite{Bao}. They suggest a novel model which combines Wavelet Transformation, SAEs 
and a Long Short Term Memory (LSTM) network. Using data from several financial exchanges (considering a 
range of ‘developed’ and ‘undeveloped’ markets), they assess the model’s applicability to OHLC prediction. 
Comparing the model to configurations without the SAE layers, and a RNN model as benchmark, they showed 
that the inclusion of SAEs resulted in less volatility and greater accuracy, which in turn offered higher profitabilities 
in a buy-and-hold trading strategy.
\hfill \break 

More novel autoencoder applications have also been attempted, with Hsu suggesting the use of a 
Recurrent Autoencoder for multidimensional time series prediction \cite{Hsu}. There is a clear pattern through the literature 
that the use of AEs and SAEs both by themselves and when used as an assisting technique result in more accurate 
prediction results and less computationally expensive training.

\section{Data Segmentation}
\hfill

One of the aspects of time series not yet discussed is how the data might be segmented for analysis. 
Financial pattern matching, as discussed in \ref{TechnicalAnalysis}, requires methodologies to decide the length of data to 
consider when determining whether a subset of data matches a pattern or not. 
\hfill\break

There are numerous classical approaches to this problem, which were widely applied in machine learning prior to 
the resurgence in deep learning. One of the earliest and more common approaches was the ‘Sliding Window’, 
where each the model input for each observation is ‘padded’ by a predetermined number of observations that 
occurred both prior and after the one in question. This has the advantage of being fairly model agnostic, but fails 
to capture any correlations in the dependent variable values \cite{Chu}. The Recurrent Sliding Window method 
aims to resolve this by including the same number of prior predictions made as part of the input. In this case, input 
would be ${<x_{t-d},...x_{t},…,x_{t+d}> and <y_{t-d},...,y_t>}$. This was shown to be significantly more effective than the plain 
sliding window method \cite{Bakiri}. Notably, the sliding window approach can only be implemented in an offline 
model.
\hfill\break

These methods can be adapted to take on online formats, as well as incorporate data reduction benefits 
through algorithms such as Piecewise Linear Approximation (PLA), which adapts a linear representation to the 
leading portion of the window. Some novel approaches (Feasible Space Window and it’s stepwise adaption) were 
suggested by Liu et al. in order to compensate for the computational requirements of reprocessing the entire 
window at each step for online models \cite{Liu}.
\hfill \break 

Window based methods represent a fairly static and unsophisticated approach to data segmentation. 
A suggested solution to this is to segment the data by dynamically identifying significant partition points, 
known as perceptual important points (PIP) \cite{Chung}. The computational cost of identifying PIPs was initially 
rather large, making them unsuitable for quickly changing and dynamic environments such as finance. 
However, Zhou et al. suggested a novel approach which reduced computational costs through intelligent binary 
tree traversal. They were able to show this approach was effective in identifying traditional financial stock patterns 
with the use of a layered neural network \cite{Zhou}. In a similar fashion, input data turning points (TP) were shown to 
potentially offer lower error rates than PIPs by Yin et al. \cite{Yin}.
\hfill \break 

More recently, Wan et al. conducted a review on the effect of segmentation on financial time series pattern 
matching, comparing perceptual important points, piecewise aggregate approximations, piecewise linear 
approximations and turning point based methods \cite{Wan}. They use several pattern matching algorithms 
(template-based, rule-based, hybrid, decision tree, and symbolic aggregate approximation) for each of the 
segmentation methods in order to determine their effect on a broader level. The analysis was performed on real 
stock market data, as well as synthetic data generated to display common patterns such as positive/negative 
head and shoulders (H&S). Measuring accuracy, precision and recall, they showed that PIP based segmentation 
generally offered superior results to the others (though it is worth noting that there were various performance 
differences within the segmentation methods depending on the matching algorithm used).








\section{Online Learning Algorithms and Gradient Descent}
\hfill

Most classic machine learning algorithms operate under the assumption that, for all intents and purposes, the 
full dataset has been collected and that the amount of training data for the model is both finite and immediately 
available. However, as the growth of information grows in an exponential fashion, there are numerous areas where
 the expected training data for the model will continue to grow. In these cases it would be disadvantageous to go 
 through the full training and validation process again in order to incorporate the newly available data.
\hfill\break

Online algorithms are designed to offset these issues by adjusting the batch training technique to rather 
repetitively draw on single samples from the data on which the model’s parameters can be adjusted. The benefit 
is that they are able to quickly process a large number of observations and readjust the model, though the 
downfall is that they are not always able to optimize the cost function to the same extent as offline batch 
algorithms \cite{Albers}.
\hfill\break

Bottou and Cun argue that as the size of the dataset grows significantly, online algorithms advantages result in 
them outperforming offline models, despite any initial drawbacks \cite{Bottou}. Previous research had shown that online 
algorithms typically perform as fast as batch algorithms during the ‘search’ phase of parameter optimization, but 
that ‘final’ phase convergence tended to fluctuate around the optima due to the noise present in single sample 
gradients \cite{LeCun, Bottou2}. Bottou and Cun showed in fact, that it is more practical to consider the convergence towards
 the parameters of the optima, rather than the optima itself (as defined by the cost function) - the difference 
 between the learning speed and optimization speed, respectively \cite{Bottou}. Theoretical and empirical findings were 
 presented to show that a stochastic online gradient descent (SGD) algorithm \textbf{[referenceappendix] }was able to 
 outperform the batch model for parameter estimation, and was able to asymptotically outperform in the number of 
 samples processed in a time period. The stochastic aspect of the algorithm is related to random observations 
 from batch sample groups being used as the gradient basis. Theoretically, this slows down the convergence, but 
 speeds up the processing speed of each batch - a technique which has later been shown to be generally 
 successful \cite{Shalev, Zhang}. 
 \hfill\break
 
 The SGD algorithm has resulted in a fair amount of further research due to its applicability to machine learning 
 and the online benefit, which can largely be group into two categories: improvements affecting gradient learning 
 and convergence rates, and processing improvements through parallelization.
\hfill\break
 
 One of the earlier improvements to convergence rates was the Momentum algorithm, as developed by Tseng \cite{Tseng}. 
 As noted, stochastic descent often introduces significant oscillation around an optima, which slows down 
 convergence. Momentum reduces this by decreasing movement in directions of high curvature, and increasing 
 increasing movement towards directions consistent with previous gradients (this is achieved through combining 
 gradient movements in opposite directions) .
\hfill\break

There have also been several attempts to introduce effective regularization into the SGD process. Bartlett et al. 
presented Adaptive Online Gradient Descent, which implements an adaptive step size through a $\lambda$ 
penalty on the learning rate, which was shown to be nearly optimal in a strong sense \cite{Bartlett}. Langford et al. 
demonstrated a variation named Truncated Gradient, which introduced an enforced weight sparsity parameter. 
The weight sparsity is able to achieve equitable effects to $L_1$ regularization (similar to Lasso Regression). 
They were able to show that implementation performed effective feature reduction, while having little effect on 
performance \cite{Langford}. Other approaches, such as AdaGrad, aim to improve the robustness of gradient training by 
adjusting the updates to parameters according to frequency - e.g. larger updates to infrequent parameters, and 
smaller updates to frequent parameters \cite{Duchi, Zeiler}. 
\hfill\break

A parallel implementation of SGD largely rests on the idea of splitting up data to be processed in individual runs 
of the gradient descent. The results are periodically aggregated, and the new model parameters are distributed 
to processing nodes for further training \cite{Zinkevich}. Depending on the configuration, the variance within results can 
result in poor convergence rates \cite{Mahajan}. Mahajan et al. suggest a novel implementation which improves the 
distribution impacts through the use of better approximating functions in the processing nodes, which in turn 
improves the efficacy of the algorithms convergence \cite{Mahajan}. Povey et al. presented a Natural Gradient SGD, 
which improves the learning rates through the use of a factor matrix used on the new gradients. They were able 
to show empirical evidence of this improving performance issues introduced from the parameter averaging 
typically used in parallelization \cite{Povey}.
\hfill\break

There have been some more recent improvements focusing on making the SGD algorithm more dynamic, 
including the Inconsistent SGD (ISGD) and AdaBatch adaptations. The idea behind ISGD is to treat the training 
on a particular batch as a stochastic process, and adjust it according to the expected loss identified. Gradient 
updates from small loss batches are relatively small compared to large loss batches, and so by focusing efforts 
here ISGD is able to optimize performance. Wang et al. performed careful testing of SGD vs. ISGD and found 
inconsistent training to consistently outperform in terms of convergence speed and results \cite{Wang}. It is worth 
noting that due to its nature, ISGD can be effectively combined with other SGD improvements (e.g. Momentum, 
as per the authors’ experiments). AdaBatch, as presented by Devarakonda et al., introduces dynamism through 
the adjustment of batch sizes. Small batch sizes in SGD offer the benefit of faster convergence in fewer training 
epochs, however larger batch sizes are more computationally efficient due to their applicability to parallelization. 
The algorithm uses a monotonically increasing batch size, which is started small to gain traction in convergence, 
and later increased to allow the benefits of data parallelism. The effect, as shown, is to offer improvements in 
training performance of up to 6x, with less than 1\% of accuracy when compared to the fixed batch baselines \cite{Devarakonda}. 
Both of these algorithms have been shown to be effective and applicable in the context of online neural network training.
\hfill\break















\section{Backtesting and Model Validation} 
\hfill

Much of financial academic literature is currently facing a problem in terms of validation and verification of results. 
The primary method of going about these ends in the past has been to perform historical simulations, or ‘backtests’ ,
in order to prove profitability of a trading strategy. The recent advances in both technology and the algorithms available 
to construct these strategies has resulted in researchers being able to run so many iterations of a model or strategy
 configuration through these backtests, that its become increasingly difficult to control for spurious results, with some 
 papers suggesting that ‘most published research findings are false’   \cite{Ioannidis}.
\hfill \break 

The standard way of implementing backtests is to split the data into two portions: an In Sample (IS) portion which
 is used to train the model, and an Out of Sample (OOS) portion which is used to test the model and validate results. 
 The problem lies in that millions of different model configurations might be tested, and if more sophisticated test 
 measures are not in place (i.e. not just the standard Neyman-Pearson hypothesis testing framework is implemented), 
 then it is only a matter of time before a false positive result occurs which shows high performance both IS and OOS (i.e. overfitting). 
 The nature of financial data, where there is a low signal-to-noise ratio in a dynamic and adaptive system, and 
 where there is only one true data sequence, makes it difficult to resolve these issues effectively 
\cite{BailyPBO, McLean}.
\hfill \break 

Overfitting is not a novel issue, and has of course been tackled in various literature areas, including machine learning. 
However, in that context, the frameworks are often not suited to the buy/sell with random frequency structure of 
investment strategies. They also do not account for overfitting outside of the output parameters, or take into 
consideration the number of trials attempted. Other methods, such as ‘hold-out’, are arguably still faulty due to researcher 
knowledge while constructing models \cite{Schorfheide}. One of the downfalls of the typical IS-OOS set up in the 
financial context is also that the most recent (and relevant) data will not be able to be used for the model training. 
\hfill \break 

There have been some suggestions to resolve the problem that is occuring in the literature as a result of this - some 
work suggesting new frameworks, which this section will cover, and others which focus on the review process or 
how data and replication procedures are made available \cite{Prado}. While the points made with regard to the review process 
and so on are certainly important, they don't aid with more effective model training for the researcher up front, and 
so will not be covered here.

\subsection{Testing Methodologies}

Considering the issues laid out above, there has been much work to develop alternative approaches to backtesting. 
One of the common approaches to avoid backtest overfitting is the ‘hold-out’ strategy, where a certain portion of 
the dataset is reserved for testing true OOS performance. Numerous problems have been pointed out with this 
approach, including that the data is often used regardless, or that awareness of the movements in the data may, 
consciously or otherwise, influence strategy and test design by the researchers \cite{Schorfheide}. For small samples, 
a hold-out strategy may be too short to be conclusive \cite{Weiss}, and even for large samples it results in the 
most recent data (which is arguably the most pertinent) not being used for model selection \cite{Hawkins, BailyPBO}.
\hfill \break 

There has been work by several authors to try and lay out techniques to try and avert backtest overfitting. 
The Model Confidence Set (MCS), as developed by Hansen et al. \cite{Hansen}, starts with a 
collection of models or configurations, and remove models iteratively according to a defined loss function. 
The confidence set is defined by the remaining models once a non-rejection takes place within the process, and 
these models are considered to be statistically similar within a certain confidence range. MCS is thus able to facilitate 
equitable model selection. However, Aparicio et al. \cite{Aparicio}, showed  that while MCS is a potential strategy, in 
practice is is ineffective due to the inordinate requirement of signal-to-noise necessary to identify true superior 
models, as well as a lack of penalization over the number of trials attempted.
\hfill \break

Bailey et a. \cite{BailyPBO} have developed a more robust approach to backtesting and how overfitting during strategy 
selection might be avoided. Their research defines backtest overfitting as having occurred when the strategy 
selection which maximizes IS performance systematically underperforms median OOS in comparison to the 
remaining configurations. They use this definition to develop a framework which measures the probability of such 
an event occuring, where the sample space is the combined pairs of IS and OOS performance of the available 
configurations. The probability of backtest overfitting (PBO) is then established as the likelihood of a configuration 
underperforming the median IS while outperforming IS. 
\hfill \break 

Formulaically, the definition of backtest overfitting is given by
\begin{equation}\label{eq:PBO1}
\sum_{n=1}^{N}E[\overline{r_n}|r\in 
\Omega_{n}^{*}]Prob[r\in\Omega_{n}^{*}]\leq{N/2}
\end{equation}

Where the search space {\textOmega} consists of the N ranked strategies, and their ranked IS performance \textit{r} and OOS performance
\textit{\={r}}. This allows the PBO, using the bayesian formula, to be defined as 

\begin{equation}\label{eq:PBO2}
  PBO = \sum_{n=1}^{N}Prob[\overline{r} < {N/2}|r\in\Omega_{n}^{*}]Prob[r\in\Omega_{n}^{*}]
  \end{equation}

Notably, the above definitions consider IS as the data made available to the strategy selection, rather than the 
models calibration (e.g. the full IS dataset, rather than, by was of example, the number of days used in a moving average). 
This allows the model-free and non-parametric nature of the definition. 
\hfill \break 

They further developed the combinatorially symmetric cross-validation (CSCV) framework as a methodology to 
reliably estimate the probability used in PBO, which allows a concrete application of the concept. The CSCV 
framework does not require using the typical ‘hold-out’ strategy (and thus avoids credibility issues), and is 
ultimately able to provide a bootstrapped distribution of OOS performance. 
\hfill \break 

The methodology can be briefly summarised (skipping some details and nuances) as the following steps:
\begin{itemize}
 \item[1]Generate a TxN performance series matrix, M, representing the profits and losses by the N trials over T time periods
\item[2]Partition the M matrix into S submatrices
\item[3]Generate the combination set C of all combinations of the S submatrices
\item[4]For each combination in C:
\begin{itemize}
\item [a] Form the training set by joining the 2 combination sets, and testing set as the rest of the combinations (all in order)
\item [b] Determine the ranked in-order IS and OOS performance for the sets
\item [c] Determine n* as the best performing IS strategy
\item [d] Determine the relative rank of the n* strategy’s OOS performance, where we should observe that \={r}* systematically outperforms OOS as 
well. Define logit \textlambda = \(\frac{\overline{\textomega}_{c}}{(1 -\overline{\textomega}_{c} )}\), where a high value implies consistency 
between IS and OOS performance, and thus a low level of backtest overfitting
\end{itemize}
\item [5] The {\textlambda} values can then be collected and used to define the relative frequency at which \textlambda 
occurs across all combination sets in C, signified by f(\textlambda).
\end{itemize}

The CSCV framework and results thus allows the consideration of several notable statistics. First and foremost, 
the PBO may now be estimated using the CSCV method and using an integral over the f(\textlambda}) function 
as defined above which offers a rate at which the best IS strategies underperform the median of OOS trials. If \textphi $\approx$ 0,
 it is evidence of no significant overfitting (inversely, \textphi  $\approx$1 would be a sign of probable overfitting). Critically then, a 
 PBO measure may be used in a standard hypothesis test to determine if a model should be rejected or not. This 
 can be extended, as shown by Bailey et al., to show the relationship between overfitting and performance 
degradation of a strategy. It becomes clear that with models overfitting to backtest data noise, there comes a point 
where seeking increased IS performance is detrimental to the goal of improving OOS performance.  
\hfill \break 

The CSCV methodology provides several important benefits (some already mentioned) over traditional testing 
frameworks, including the usual K-fold cross validation used in machine learning. By recombining the slices of 
available data, both the training and testing sets are of equal size, which is particularly advantageous when comparing 
financial statistics such as the Sharpe Ratio (SR), which are susceptible to sample size. Additionally, the symmetry 
of the set combinations in CSCV ensure that performance degradation is only as a result of overfitting, and not 
arbitrary differences in data sets. It is pointed out that while CSCV and PBO should be used to evaluate the quality 
of a strategy, they should not be the function on which strategy selection relies, which in itself would result in overfitting.
\hfill \break 

\subsection{Test Data Length}

The CSCV methodology offers an important but highly generalised framework to assess models and backtest 
overfitting. It doesn’t however indicate which metrics should be used to assess the IS and OOS performance, nor 
any indication on the amount of data needed to do so effectively. One of the noted limitations of the framework is 
that a high PBO indicates overfitting within the group of N strategies, which is not necessarily indicative that none 
of the strategies are skillful - it could be that all of them are. Also, as pointed out, it should not be used as an 
objective function to avoid overfitting, but rather as an evaluation tool. To this end it helps assess overfitting, but 
not necessarily avoid it. 
\hfill \break 

A typical measure of evaluation used for financial models is the Sharpe Ratio (SR), which is the ratio of between 
average excess returns and the returns’ standard deviation - a measure of the return on risk. In the context of 
comparing models, SR is typically expressed annually to allow models with different frequencies to be compared. 
Lo et al. \cite{Lo} show that annulaized SR can be expressed as

\begin{equation}\label{SRAnnual}
SR=\frac{\mu}{\sigma}\sqrt{q}
\end{equation}

Using sample means and deviations, $\hat{\mu}$ and $\hat{\sigma}$, SR can be shown to converge as follows 
(as y $\rightarrow\infty$})

\begin{equation}\label{SRConvergence}
  \hat{SR}  \rightarrow \mathcal {N} [SR,\frac{1 + \frac{SR^2}{2q}}{y}]
\end{equation}

Thus, when using SR estimations, which follow a Normal distribution, it is possible that where the true SR mean is 
zero we may still (with enough configurations attempted) find an SR measurement which optimises IS performance. 
This is shown by Bailey et al. \cite{BaileyBTL}, who propose the non-null probability of selecting an IS strategy with null expected 
performance OOS. Notably, typical methods such as hold-out once again fail, as the number of configurations 
attempted are not recorded. They add a further derivation, thich is the Minimum Backtest Length (MinBTL), ultimately 
showing that

\begin{equation}\label{MinBTL}
MinBTL \approx (\frac{
                                  (1-\gamma)Z^{-1}[1-\frac{1}{N}] + \gamma Z^{-1}[1 -\frac{1}{N}e^{-1}]}
                                  {\overline{E[max_N]}})^2
                                  < \frac{2ln[N]}{\overline{E[max_N]}}^2
\end{equation}

The statistic highlights the relationships between: selecting a strategy with a higher IS SR than expected OOS, 
the number of strategies tested (N), and the number of years tested (y). The equation shows that  as the number 
of strategies tested increases, the minimum back test length much also increase in order to contain the likelihood 
of overfitting to IS SR. 
\hfill \break 

As shown extensively throughout ML literature, increased model complexity and number of parameters is one of 
the primary causes of overfitting. In context of the MinBTL formula, model complexity affects the number of 
configurations that are available and which may be tested, which in turn will increase likelihood of overfitting. 
A lack of consideration, or reporting, of the number of trials makes the potential for overfitting impossible to assess. 
\hfill \break 

Bailey et al. expanded on this view with assessing the impact of presenting overfit models as correct. 
They were able to show that in lieu of any compensation effects (i.e. a series following a Gaussian random walk), 
there is no reason for overfitting to result in negative performance. However, where compensation effects apply 
(e.g. economic/investment cycles, bubble bursts, major corrections etc.), then the inclusion of memory in a strategy
 is likely to be detrimental to OOS performance if overfitting isn’t controlled for \cite{BaileyBTL}.
\hfill \break 

\subsection {Sharpe Ratio}

The use of the Sharpe Ratio in financial backtesting is not just an arbitrary or persistent literature choice. 
The statistic offers two benefits: the effectively strategy-agnostic financial information contained, as well as being 
relatable to the t-statistic, and so simple to perform hypothesis testing. The SR ratio (estimate from sample as $\hat{SR}$) 
is defined as

\begin{equation}\label{SR}
  SR=\frac{\mu}{\sigma}
\end{equation}

The t-ratio is defined as 

\begin{equation}\label{tratio}
  t-ratio = \frac{\hat{\mu}}{\hat{\sigma}/\sqrt{T}}
\end{equation}

Evidently, the link here is trivial, as per formula \eqref{SRAnnual} . As noted earlier though, the chances of 
overfitting with the SR ratio, even if true mean returns are zero, are relatively significant. One of the strategies 
employed to try and counteract this is to use a 'haircut', where the reported SR ratio is discounted by 50\%. 
\hfill \break 

The 50\% however, is merely a rule of thumb, and Harvey et Lui \cite{Harvey} were able to report significant work showing 
that in a context of multiple testing, the haircut is nonlinear - the highest Sharpe ratios are moderately penalized, 
whereas the marginal Sharpe ratios were heavily penalized. While initially fairly sensible, Harvey et al raise valid 
concerns regarding the effect on option strategies, controlling for risk as well as, pertinently, what constitutes an 
appropriate level of significance testing. In light of this, they develop a p-value based statistic for multiple testing, 
the haircut adjusted sharpe ratio HSR, as well as expand upon work by Harvey et al. \cite{HLZ} to provide a distribution that can 
be used in a dependent multiple testing framework with an appropriate p-value adjustment.
\hfill \break 

This work is relevant, in that the HSR statistics proposed offer another framework in which investment strategies 
might be evaluated against each other. The primary difference in comparison to PBO and CSCV, is that where 
they offer a methodology for evaluating strategies within a group, HSR aims to adjust the statistical significance 
of each strategy such that the overall risk of spurious correlation is controlled for. A benefit of this method is that 
there is less chance of a relevant strategy being disregarded as a result of just poor peer performance. PBO 
however, does have the primary benefit of being metric-agnostic, where the HSR framework is largely based on 
using the Sharpe ratio (though it can be generalized to another statistic with a probabilistic interpretation). Additionally, 
PBO is generally more in line with machine learning literature with the cross validation like approach on time series data.  
\hfill \break 

It should be noted, that the literature detailing usage of the Sharpe ratio for strategy comparison is extensive, with 
numerous variations and methodologies offered \cite{BaileySharpe}. However, the crux of this paper lies 
in whether an online neural network is able to make effective enough predictions that a strategy might use the 
predictions to be profitable. The subtlety here is that we will consider the usage of such forecasting \textit{within} a strategy,
 rather than \textit{as} a strategy itself. In line with this, statistics such as the Sharpe ratio will be used, but not form a critical 
 consideration of the research here as the comparison of strategies used will be a secondary concern.
\hfill \break 

\newpage
\begin{thebibliography}{9}

\bibitem{Albers}
Albers S. Online Algorithms: a survey. Mathematics Subject Classification (1991): 68W25, 
68W40. Available at: https://link.springer.com/content/pdf/10.1007%2Fs10107-003-0436-0.pdf

\bibitem{Aparicio}
Aparicio, Diego and Lopez de Prado, Marcos, How Hard Is It to Pick the Right Model? (December 2017). Available at SSRN: https://ssrn.com/abstract=3044740 or http://dx.doi.org/10.2139/ssrn.3044740

\bibitem{BailyPBO}
Bailey, David H. and Borwein, Jonathan and Lopez de Prado, Marcos and Zhu, Qiji Jim, The Probability of Backtest Overfitting (February 27, 2015). Journal of Computational Finance (Risk Journals), 2015, Forthcoming. Available at SSRN: https://ssrn.com/abstract=2326253 or http://dx.doi.org/10.2139/ssrn.2326253

\bibitem{BaileyBTL}
Bailey, David H. and Borwein, Jonathan and Lopez de Prado, Marcos and Zhu, Qiji Jim, Pseudo-Mathematics and Financial Charlatanism: The Effects of Backtest Overfitting on Out-of-Sample Performance (April 1, 2014). Notices of the American Mathematical Society, 61(5), May 2014, pp.458-471. Available at SSRN: https://ssrn.com/abstract=2308659 or http://dx.doi.org/10.2139/ssrn.2308659

\bibitem{BaileySharpe}
Bailey, David H. and Lopez de Prado, Marcos, The Deflated Sharpe Ratio: Correcting for Selection Bias, Backtest Overfitting and Non-Normality (July 31, 2014). Journal of Portfolio Management, 40 (5), pp. 94-107. 2014 (40th Anniversary Special Issue).. Available at SSRN: https://ssrn.com/abstract=2460551 or http://dx.doi.org/10.2139/ssrn.2460551

\bibitem{Bakiri}
G. Bakiri and T. G. Dietterich. Achieving high-accuracy text-to-speech with machine learning. In R. I. Damper, editor, Data Mining Techniques in Speech Synthesis. Chapman and Hall, New York, NY, 2002. 22

\bibitem{Bao}
Bao W, Yue J, Rao Y (2017) A deep learning framework for financial time series using stacked autoencoders and long-short term memory. PLoS ONE 12(7): e0180944. https://doi.org/10.1371/journal.pone.0180944

\bibitem{Bartlett}
Bartlett P., Hazan E. Adaptive Online Gradient Descent. Available at: http://papers.nips.cc/paper/3319-adaptive-online-gradient-descent.pdf

\bibitem{Bengio1}
Yoshua Bengio, Pascal Lamblin, Dan Popovici, and Hugo Larochelle. Greedy layer-wise training of deep networks. In Bernhard Scholkopf, John Platt, and Thomas Hoffman, editors, ¨ Advances in Neural Information Processing Systems 19 (NIPS’06), pages 153–160. MIT Press, 
2007. Available at: http://papers.nips.cc/paper/3048-greedy-layer-wise-training-of-deep-networks.pdf

\bibitem{Bottou}
Bottou L., Le Cun Y. Large Scale Online Learning. Available at: http://papers.nips.cc/paper/2365-large-scale-online-learning.pdf

\bibitem{Bottou2}
Bottou, L. and Murata, N. (2002). Stochastic Approximations and Efficient Learning. In Arbib, M. A., editor, The Handbook of Brain Theory and Neural Networks, Second edition,. The MIT Press, Cambridge, 

\bibitem{Chu}
Chu C., Time series segmentation: A sliding window approach. Information Sciences, Volume 85, Issues 1–3, July 1995, Pages 
147-173. Available at: https://www.sciencedirect.com/science/article/pii/002002559500021G#!

\bibitem{Chung}
 F. L. Chung, T. C. Fu, R. Luk, and V. Ng, “Flexible time series pattern matching based on perceptually important points,’’ in Proc. Int. Joint Conf. Artificial Intelligence Workshop  
 Learning from Temporal and Spatial Data in International Joint Conference on Artificial Intelligence (IJCAI'01), Seattle, Washington, USA, 4-10 August, 2001, p. 
 1-7. Available at: http://hdl.handle.net/10397/48578

\bibitem{Devarakonda}
Aditya Devarakonda, Maxim Naumov & Michael Garland. ADABATCH: ADAPTIVE BATCH SIZES FOR TRAINING
DEEP NEURAL NETWORKS. Available at: https://arxiv.org/pdf/1712.02029.pdf

\bibitem{Donoho}
David L. Donoho, High-Dimensional Data Analysis: The Curses and Blessings of 
Dimensionality (August 8, 2000). Available at: http://citeseerx.ist.psu.edu/viewdoc/download?doi=10.1.1.329.3392&rep=rep1&type=pdf

\bibitem{Duchi}
John Duchi, Elad Hazan and Yoram Singer. Adaptive Subgradient Methods for
Online Learning and Stochastic Optimization. Journal of Machine Learning Research 12 (2011) 
2121-2159. Available at: http://www.jmlr.org/papers/volume12/duchi11a/duchi11a.pdf

\bibitem{Erhan}
Erghan D., Bengio Y., Courville A, Manzagol P., Vincent P. Why Does Unsupervised Pre-training Help Deep 
Learning?, Journal of Machine Learning Research 11 (2010) 625-660 . Available 
at: http://www.jmlr.org/papers/volume11/erhan10a/erhan10a.pdf

\bibitem{Fama}
E.F. Fama, The behavior of stock-market prices, J. Bus., 1 (1965), pp. 34-105. 
Available at: https://doi.org/10.1086/294743

\bibitem{Fan1}
Jianqing Fan, Runze Li, 
Statistical Challenges with High Dimensionality: Feature Selection in Knowledge Discovery 
(7 Feb, 2006). Available at: https://arxiv.org/abs/math/0602133

\bibitem{Fan2}
Fan, J., & Fan, Y. (2008). High Dimensional Classification Using Features Annealed Independence Rules. Annals of Statistics, 36(6), 2605–2637. http://doi.org/10.1214/07-AOS504

\bibitem{Hansen}
Hansen, Peter Reinhard and Lunde, Asger and Nason, James M., The Model Confidence Set (March 18, 2010). Available at SSRN: https://ssrn.com/abstract=522382 or http://dx.doi.org/10.2139/ssrn.522382

\bibitem{Harvey}
Harvey, Campbell R. and Liu, Yan, Backtesting (July 28, 2015). Available at SSRN: https://ssrn.com/abstract=2345489 or http://dx.doi.org/10.2139/ssrn.2345489

\bibitem{Hawkins}
Hawkins, Douglas. (2004). The Problem of Overfitting. Journal of chemical information and computer sciences. 44. 1-12. 10.1021/ci0342472. 

\bibitem{Hinton1}
Geoffrey E. Hinton, Simon Osindero, and Yee Whye Teh. A fast learning algorithm for deep belief nets. Neural Computation, 18:1527–1554, 2006. 
Available at: https://www.mitpressjournals.org/doi/abs/10.1162/neco.2006.18.7.1527

\bibitem{Hinton2}
G. E. Hinton, R. R. Salakhutdinov, Reducing the Dimensionality of Data with Neural 
Networks. Science  28 Jul 2006: Vol. 313, Issue 5786, pp. 504-507 DOI: 
10.1126/science.1127647. Available at: http://science.sciencemag.org/content/313/5786/504/tab-pdf

\bibitem{Hinton3}
Geoffrey E. Hinton, Training Products of Experts by Minimizing Contrastive Divergence,Neural Computation
Volume 14 | Issue 8 | August 2002  p.1771-1800 . Available at: https://www.mitpressjournals.org/doi/pdf/10.1162/089976602760128018

\bibitem{HLZ}
Campbell R. Harvey & Yan Liu & Heqing Zhu, 2016. "… and the Cross-Section of Expected Returns," Review of Financial Studies, vol 29(1), pages 5-68.

\bibitem {Hornik}
K. Hornik, Multilayer feed-forward networks are universal approximators, Neural Networks, vol 2, 1989

\bibitem{Hsu}
Hsu D. Time Series Compression Based on Adaptive Piecewise
Recurrent Autoencoder (17 August 2017). Available at: https://arxiv.org/pdf/1707.07961.pdf

\bibitem{Ioannidis} 
Ioannidis JPA (2005) Why Most Published Research Findings Are False. PLoS Med 2(8): e124. https://doi.org/10.1371/journal.pmed.0020124

\bibitem{Langford}
John Langford, Lihong Li and Tong Zhang. Sparse Online Learning via Truncated 
Gradient. Journal of Machine Learning Research 10 (2009) 777-801 Submitted 6/08; Revised 11/08; Published 
3/09. Available at: http://www.jmlr.org/papers/volume10/langford09a/langford09a.pdf 

\bibitem{Langkvist}
Martin Längkvist, Lars Karlsson,  Amy Loutfi, A review of unsupervised feature learning and deep learning for time-series 
modeling, Applied Autonomous Sensor Systems, School of Science and Technology, Örebro University, SE-701 82 Örebro, 
Sweden. Available at: https://www.sciencedirect.com/science/article/pii/S0167865514000221#bi005

\bibitem{LeCun}
LeCun Y., Bottou L, Orr G, Muller K. Efficient Backprop. Available at: http://yann.lecun.com/exdb/publis/pdf/lecun-98b.pdf

\bibitem{Liu}
Liu X., Lin Z., Wang H. Novel Online Methods for Time Series Segmentation.  IEEE TRANSACTIONS ON KNOWLEDGE AND DATA ENGINEERING, VOL. 20, NO. 12, DECEMBER 
2008. Available at: https://ieeexplore.ieee.org/stamp/stamp.jsp?arnumber=4445667&tag=1

\bibitem{Lo}
Lo, Andrew W., The Statistics of Sharpe Ratios. Financial Analysts Journal, Vol. 58, No. 4, July/August 2002. Available at SSRN: https://ssrn.com/abstract=377260

\bibitem{Lv}
Yisheng Lv, Yanjie Duan, Wenwen Kang, Zhengxi Li, and Fei-Yue Wang. Traffic Flow Prediction With Big Data:
A Deep Learning Approach. IEEE TRANSACTIONS ON INTELLIGENT TRANSPORTATION SYSTEMS, VOL. 16, NO. 2, APRIL 
2015. Available at: https://ieeexplore.ieee.org/stamp/stamp.jsp?tp=&arnumber=6894591&tag=1

\bibitem{Mahajan}
Mahajan D., Keerthi S., Sundararajan S., Bottou L. A Parallel SGD method with Strong 
Convergence. Available at: https://arxiv.org/pdf/1311.0636.pdf

\bibitem{McLean}
McLean, R. David and Pontiff, Jeffrey, Does Academic Research Destroy Stock Return Predictability? (January 7, 2015). Journal of Finance, Forthcoming. Available at SSRN: https://ssrn.com/abstract=2156623 or http://dx.doi.org/10.2139/ssrn.2156623

\bibitem{Prado}
Lopez de Prado, Marcos, The Future of Empirical Finance (May 31, 2015). Journal of Portfolio Management, 41(4). Summer 2015. Forthcoming.. Available at SSRN: https://ssrn.com/abstract=2609734 or http://dx.doi.org/10.2139/ssrn.2609734

\bibitem{Povey}
Povey D., Zhang X., Khudanpur S. Parallel training of Deep Neural Networks with Natural Gradient
and Parameter Averaging. Available at: http://citeseerx.ist.psu.edu/viewdoc/download?doi=10.1.1.745.6995&rep=rep1&type=pdf

\bibitem{Ranzato1}
Marc’Aurelio Ranzato, Christopher Poultney, Sumit Chopra, and Yann LeCun. Efficient learning of sparse representations with an energy-based model. In B. Scholkopf, J. Platt, and T. Hoffman, ¨ editors, Advances in Neural Information Processing Systems 19 (NIPS’06), pages 1137–1144. MIT Press, 
2007. Aavilable at: http://papers.nips.cc/paper/3112-efficient-learning-of-sparse-representations-with-an-energy-based-model.pdf

\bibitem{Schaefer}
Robert L. Schaefer (2012) Subset Selection in Regression, Technometrics, 34:2, 229, DOI: 10.1080/00401706.1992.10484917
  
\bibitem{Schorfheide}
Schorfheide, Frank, and Kenneth I. Wolpin. 2012. "On the Use of Holdout Samples for Model Selection." American Economic Review, 102 (3): 477-81.

\bibitem{Shalev}
Shai Shalev-Shwartz, Yoram Singer, and Nathan Srebro. Pegasos: Primal Estimated sub-GrAdient SOlver for SVM. In Proceedings of the Twenty-Fourth International Conference on Machine Learning (ICML-07), 2007.

\bibitem{Takeuchi}
Takeuchi L, Lee Y. Applying Deep Learning to Enhance Momentum Trading Strategies in 
Stocks. Technical Report, 2013. Available at: http://www.smallake.kr/wp-content/uploads/2017/04/TakeuchiLee-ApplyingDeepLearningToEnhanceMomentumTradingStrategiesInStocks.pdf

\bibitem{Troiano}
Troiano L., Mejuto E., Kriplani P.On Feature Reduction using Deep Learning
for Trend Prediction in Finance (11 Apr 2017).  Available at: 
https://arxiv.org/abs/1704.03205.

\bibitem{Tseng}
Tseng P., AN INCREMENTAL GRADIENT(-PROJECTION) METHOD
WITH MOMENTUM TERM AND ADAPTIVE STEPSIZE RULE. SIAM J. OPTIM.  Vol. 8, No. 2, pp. 506–531, May 
1998. Available at: https://pdfs.semanticscholar.org/1a29/6a1577478654a54a9f801f93f71b7d853c53.pdf

\bibitem{Vincent}
Vincent P., Larochell H., Lajoie I., Bengio Y., Manzagol P., Stacked Denoising Autoencoders: Learning Useful Representations in
a Deep Network with a Local Denoising Criterion. Journal of Machine Learning Research 11 (2010) 
3371-3408. Available at: http://www.jmlr.org/papers/volume11/vincent10a/vincent10a.pdf

\bibitem{Wan}
Wan Y., Gong X., Si Y. Effect of segmentation on financial time series pattern 
matching. Applied Soft Computing Volume 38, January 2016, Pages 346-359. 
Available at: https://www.sciencedirect.com/science/article/pii/S1568494615006341

\bibitem {Wang}
Linnan Wang,  Yi Yang, Renqiang Min, Srimat Chakradhar. Accelerating deep neural network training with 
inconsistent stochastic gradient descent. Neural Networks
Volume 93, September 2017, Pages 219-229. Available at: https://www.sciencedirect.com/science/article/pii/S0893608017301399

\bibitem{Weiss}
Weiss, S. M, & Kulikowski, C. A. (1991). Computer systems that learn : classification and prediction methods from statistics, neural nets, machine learning, and expert systems. San Mateo (Calif.): Kaufmann.

\bibitem{Yin}
Yin J., Si Y., Gong Z. Financial Time Series Segmentation Based On Turning 
Points. Proceedings of 2011 International Conference on System Science and Engineering, Macau, China - June 
2011. Available at: https://ieeexplore.ieee.org/stamp/stamp.jsp?tp=&arnumber=5961935

\bibitem{Zeiler}
Zeiler M. ADADELTA: An Adaptive Learning Rate Method. Available at: https://arxiv.org/abs/1212.5701

\bibitem{Zinkevich}
Zinkevich M, Weimer M, Smola A, Li L. Parallelized Stochastic Gradient Descent. 
Available at: 
http://papers.nips.cc/paper/4006-parallelized-stochastic-gradient-descent.pdf

\bibitem{Zhao}
Zhao Y., Li J., Yu L. A deep learning ensemble approach for crude oil price 
forecasting.  Available at: https://doi.org/10.1016/j.eneco.2017.05.023

\bibitem{Zhang}
Tong Zhang. Solving large scale linear prediction problems using stochastic gradient descent algorithms. In Proceedings of the Twenty-First International Conference on Machine Learning (ICML-04), pages 919–926, 2004

\bibitem{Zhou}
Zhou B., Hu J. A Dynamic Pattern Recognition Approach Based on Neural Network for Stock 
Time-Series. 2009 World Congress on Nature & Biologically Inspired Computing (NaBIC 
2009). Available at: https://ieeexplore.ieee.org/stamp/stamp.jsp?tp=&arnumber=5393674

\end{thebibliography}
\end {document}


